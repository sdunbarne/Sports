
\begin{description}

% \item[Geogebra] 

% \link{  .ggb}{GeoGebra applet}

\item[R] 

\link{http://www.math.unl.edu/~sdunbar1/Teaching/Probability/MarkovChains/Sports/tennis.R}{R script for tennis.}

\begin{lstlisting}[language=R]
library("markovchain")

rows <- 17

p <- 0.51
q <- 1-p

P <- matrix(0, rows, rows)
P[1,  2] <- q; P[1,3] <- p;
P[2,  4] <- q; P[2,5] <- p;
P[3,  5] <- q; P[3,6] <- p;
P[4,  7] <- q; P[4,8] <- p;
P[5,  8] <- q; P[5,9] <- p;
P[6,  9] <- q; P[6,10] <- p;
P[7, 16] <- q; P[7,11] <- p;
P[8, 11] <- q; P[8,14] <- p;
P[9, 14] <- q; P[9,12] <- p;
P[10,12] <- q; P[10,17] <- p;
P[11,16] <- q; P[11,13] <- p;
P[12,15] <- q; P[12,17] <- p;
P[13,16] <- q; P[13,14] <- p;
P[14,13] <- q; P[14,15] <- p;
P[15,14] <- q; P[15,17] <- p
P[16,16] <- 1;
P[17,17] <- 1;

tennis  <- new("markovchain", transitionMatrix=P, name="Tennis")
print( summary(tennis) )

cat("Probability A wins game: ",
    absorptionProbabilities(tennis)[1,2], "\n")

T = P[1:15,1:15]
eye = diag(15)
N = solve(eye-T, eye)
ones = matrix(1, 15,1)
VarW = (2*N-eye) %*% N %*% ones - (N %*% ones)^2

cat("Expected length of game in serves: ",
      meanAbsorptionTime(tennis)[1],
    "Standard deviation: ", sqrt(VarW[1]), "\n")
\end{lstlisting}  


\link{http://www.math.unl.edu/~sdunbar1/Teaching/Probability/MarkovChains/Sports/bball.R}{R script for basketball.}
\begin{lstlisting}
library("markovchain")

rows <- 8

peffA <- 0.6
peffB <- 0.75

pdA <-  1-peffB
pdB <- 1 - peffA

prA <- 0.3
prB <- 0.3

psA <- 0.7
psB <- 0.6

stateNames <- c("AOff", "APen", "AScore", "Bdef",
                "BOff", "BPen", "BScore", "ADef")
P <- matrix(c(0,   peffA, 0,   1-peffA,   0,     0, 0, 0,
              0,   prA,   psA, 1-prA-psA, 0,     0, 0, 0,
              0,   0,     0,   0,         1,     0, 0, 0,
              pdA, 0,     0,   0,         1-pdA, 0, 0, 0,
              0,   0,     0,   0,         0,     peffB, 0,   1-peffB,
              0,   0,     0,   0,         0,     prB,   psB, 1-prB-psB,
              1,   0,     0,   0,         0,     0, 0, 0,
              1-pdB,0,    0,   0,         pdB,   0, 0, 0),
              nrow=rows, byrow=TRUE)
rownames(P) <- stateNames
colnames(P) <- stateNames

startState  <- "AOff"

bball  <- new("markovchain", transitionMatrix=P,
                 states=stateNames, name="Basketball")
print( summary(bball) )

pathLength  <- 450
ngames <- 100
allAScores <- numeric(ngames)
allBScores <- numeric(ngames)

for (i in 1:ngames) {
    bballHistory  <- rmarkovchain(n=pathLength, object=bball,
                                 t0=startState)
    allAScores[i] <- 2 * sum(bballHistory == "AScore")
    allBScores[i] <- 2 * sum(bballHistory == "BScore")
}

cat("Mean A Scores: ",      mean(allAScores),
    "Standard Deviation: ", sd(allAScores), "\n")
cat("Mean B Scores: ",      mean(allBScores),
    "Standard Deviation: ", sd(allBScores), "\n")

\end{lstlisting}
% \item[Octave]

% \link{http://www.math.unl.edu/~sdunbar1/    .m}{Octave script for .}

% \begin{lstlisting}[language=Octave]

% \end{lstlisting}

% \item[Perl] 

% \link{http://www.math.unl.edu/~sdunbar1/    .pl}{Perl PDL script for .}

% \begin{lstlisting}[language=Perl]

% \end{lstlisting}

% \item[SciPy] 

% \link{http://www.math.unl.edu/~sdunbar1/    .py}{Scientific Python script for .}

% \begin{lstlisting}[language=Python]

% \end{lstlisting}

\end{description}



% %%% Local Variables:
% %%% TeX-master: t
% %%% End:
