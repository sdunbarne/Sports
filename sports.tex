%%% -*-LaTeX-*-
%%% sports.tex.orig
%%% Prettyprinted by texpretty lex version 0.02 [21-May-2001]
%%% on Tue Jul  5 08:46:25 2022
%%% for Steve Dunbar (sdunbar@family-desktop)
%%% [[id:acfe42d6-c566-470a-93ef-673087cf5f87][Sports]]

\documentclass[12pt]{article}

\input{../../../../etc/macros} % \input{../../../../etc/mzlatex_macros}
\input{../../../../etc/pdf_macros}

\bibliographystyle{plain}

\begin{document}

\myheader \mytitle

\hr

\sectiontitle{Modeling Sports With Markov Chains}

\hr

\usefirefox

\hr

% \visual{Study Tip}{../../../../CommonInformation/Lessons/studytip.png}
% \section*{Study Tip}

% \hr

\visual{Rating}{../../../../CommonInformation/Lessons/rating.png}
\section*{Rating} %one of
% Everyone: contains no mathematics.
Student:  contains scenes of mild algebra or calculus that may require
guidance.
% Mathematically Mature: may contain mathematics beyond calculus with proofs.
% Mathematicians Only: prolonged scenes of intense rigor.

\hr

\visual{Section Starter Question}{../../../../CommonInformation/Lessons/question_mark.png}
\section*{Section Starter Question}

What is your favorite sport?  Does the sport have discrete ``states''
that can be modeled with a Markov chain?  What sports analytic measures
exist to measure the game?

\hr

\visual{Key Concepts}{../../../../CommonInformation/Lessons/keyconcepts.png}
\section*{Key Concepts}

\begin{enumerate}
    \item
        Tennis can be modeled as a Markov chain with two absorbing
        states.
    \item
        Basketball can be modeled as a Markov chain.  The sample path of
        the chain predicts the final score of the game.
\end{enumerate}

\hr

\visual{Vocabulary}{../../../../CommonInformation/Lessons/vocabulary.png}
\section*{Vocabulary}
\begin{enumerate}
    \item
    \item
\end{enumerate}

\hr

\section*{Notation}
\begin{enumerate}
    \item
\end{enumerate}

\visual{Mathematical Ideas}{../../../../CommonInformation/Lessons/mathematicalideas.png}
\section*{Mathematical Ideas}

Sports provide examples of Markov chains with \( 8 \) to \( 20 \)
states.  For some sports, winning or losing are absorbing states of the
Markov chain.  In other cases, the sample path of the chain provides a
score for the game.  The examples illustrate previously derived methods
for Markov chains.

\subsection*{Tennis}

One player in a game is the server, and the opposing player is the
receiver.  A coin toss determines the server in the first game. Service
alternates game by game between the two players or teams.

A tennis game consists of a sequence of points played with the same
player serving.  The first player to gain \( 4 \) points in total and at
least two points more than the opponent wins.  The scoring of tennis is
unusual for historical reasons that are not clear.  Roughly speaking, a
player's point score advances by \( 15 \) when a player scores a point
because the opponent missed returning the ball or hits it out of bounds,
with a few more technical rules.  The maximum score to win a game is \(
60 \) so game scores should be \( 0 \), \( 15 \), \( 30 \), \( 45 \),
and \( 60 \).  But tradition abbreviates the third score to \( 40 \),
and calls the \( 0 \) score ``love'' for reasons that are again
historically unclear.  Thus, the running score of each game is special
to tennis:  scores from zero to three points are ``love'', ``15'',
``30'', and ``40'', respectively.  If each player has at least three
points, making the player's scores equal at 40, the score is not called
out as ``40--40'', but rather as ``deuce''.  If at least three points
have been scored by each side and a player has one more point than the
opponent, called ``ad in'' when the serving player is ahead, and ``ad
out'' when the receiving player is ahead.

To model tennis as a Markov chain, take the states of the chain as the
possible scores.  For simplicity, assume that transition from one state
or score to another is constant and depends only on the present score
and not on the preceding states or scores. Figure~%
\ref{fig:sports:tennis} illustrates the transitions among the states.
Clearly this assumption is idealizing the game, leaving out factors such
as the order of service, psychological factors, fatigue, or adaptation
to the opponent's style. Other factors not mentioned can also affect the
transition probabilities.%
\index{tennis}

\begin{figure}
    \centering
\begin{asy}
  settings.outformat = "pdf";

// import graph;

size(5inches);

real myfontsize = 12;
real mylineskip = 1.2*myfontsize;
pen mypen = fontsize(myfontsize, mylineskip);
defaultpen(mypen);

real marge = 0mm;

real h = 1, v = 1;		// horiz and vert spacing

pair r05=(0,5*v);
pair rm14=(-0.75*h,4*v), r14=(0.75*h,4*v);
pair rm13=(-1.5*h, 3*v), r03=(0,3*v), r13=(1.5*h, 3v);
pair rm22=(-2*h, 2*v), rm12=(-0.75*h, 2*v), r12=(0.75*h, 2*v), r22=(2*h, 2*v);
pair rm11=(-1.5*h,v), r11=(1.5*h,v);
pair rm20=(-3*h, 0), rm10=(-1.5*h,0), r00=(0,0), r10=(1.5*h,0), r20=(3*h,0);

object state00=draw("(1) 0:0", roundbox, r05, marge);

object
state015=draw("(2) 0:15", roundbox, rm14, marge),
state150=draw("(3) 15:0", roundbox, r14, marge);

object
state030=draw("(4) 0:30", roundbox, rm13, marge),
state1515=draw("(5) 15:15", roundbox, r03, marge),
state300=draw("(6) 30:0", roundbox, r13, marge);

object
state040=draw("(7) 0:40", roundbox, rm22, marge),
state1530=draw("(8) 15:30", roundbox, rm12, marge),
state3015=draw("(9) 30:15", roundbox, r12, marge),
state400=draw("(10) 40:0", roundbox, r22, marge);

object
state1540=draw("(11) 15:40", roundbox, rm11, marge),
state4015=draw("(12) 40:15", roundbox, r11, marge);

object
stateBwins=draw("(16) B wins", roundbox, rm20, marge),
state3040=draw("(13) 30:40", roundbox, rm10, marge),
state3030=draw("(14) 30:30", roundbox, r00, marge),
state4030=draw("(15) 40:30", roundbox, r10, marge),
stateAwins=draw("(17) A wins", roundbox, r20, marge);

add(new void(picture pic, transform t) {
draw(pic, Label("$p$"),
point(state00,S,t)--point(state150,N,t),Arrow);
draw(pic, Label("$q$"),
point(state00,S,t)--point(state015,N,t),Arrow);

draw(pic, Label("$p$"),
point(state150,S,t)--point(state300,N,t),Arrow);
draw(pic, Label("$q$"),
point(state150,S,t)--point(state1515,N,t),Arrow);

draw(pic, Label("$p$"),
point(state015,S,t)--point(state1515,N,t),Arrow);
draw(pic, Label("$q$"),
point(state015,S,t)--point(state030,N,t),Arrow);

draw(pic, Label("$p$"),
point(state015,S,t)--point(state1515,N,t),Arrow);
draw(pic, Label("$q$"),
point(state015,S,t)--point(state030,N,t),Arrow);

draw(pic, Label("$p$"),
point(state300,S,t)--point(state400,N,t),Arrow);
draw(pic, Label("$q$"),
point(state300,S,t)--point(state3015,N,t),Arrow);

draw(pic, Label("$p$"),
point(state1515,S,t)--point(state3015,N,t),Arrow);
draw(pic, Label("$q$"),
point(state1515,S,t)--point(state1530,N,t),Arrow);

draw(pic, Label("$p$"),
point(state030,S,t)--point(state1530,N,t),Arrow);
draw(pic, Label("$q$"),
point(state030,S,t)--point(state040,N,t),Arrow);

draw(pic, Label("$p$"),
point(state400,S,t)--point(stateAwins,N,t),Arrow);
draw(pic, Label("$q$"),
point(state400,S,t)--point(state4015,N,t),Arrow);

draw(pic, Label("$p$"),
point(state3015,S,t)--point(state4015,N,t),Arrow);
draw(pic, Label("$q$"),
point(state3015,S,t)--point(state3030,N,t),Arrow);

draw(pic, Label("$p$"),
point(state1530,S,t)--point(state3030,N,t),Arrow);
draw(pic, Label("$q$"),
point(state1530,S,t)--point(state1540,N,t),Arrow);

draw(pic, Label("$p$"),
point(state040,S,t)--point(state1540,N,t),Arrow);
draw(pic, Label("$q$"),
point(state040,S,t)--point(stateBwins,N,t),Arrow);


draw(pic, Label("$p$"),
point(state4015,S,t)--point(stateAwins,N,t),Arrow);
draw(pic, Label("$q$"),
point(state4015,S,t)--point(state4030,N,t),Arrow);

draw(pic, Label("$p$"),
point(state1540,S,t)--point(state3040,N,t),Arrow);
draw(pic, Label("$q$"),
point(state1540,S,t)--point(stateBwins,N,t),Arrow);

draw(pic, Label("$p$", align=N),
point(state4030,E,t)--point(stateAwins,W,t),Arrow);

draw(pic, Label("$p$", align=N),
point(state3030,NE,t)--point(state4030,NW,t),Arrow);
draw(pic, Label("$q$", align=S),
point(state4030,SW,t)--point(state3030,SE,t),Arrow);

draw(pic, Label("$p$", align=N),
point(state3040,NE,t)--point(state3030,NW,t),Arrow);
draw(pic, Label("$q$", align=S),
point(state3030,SW,t)--point(state3040,SE,t),Arrow);

draw(pic, Label("$q$", align=S),
point(state3040,W,t)--point(stateBwins,E,t),Arrow);
  });
\end{asy}
    \caption{The state transition diagram for a game of tennis.  The
    score for A is first, separated by a colon from the score for B. The
    numbers in parentheses are the row and column numbers of the state
    in the probability transition matrix.  The probability labels are
    the same as the analogous random walk probabilities.}%
    \label{fig:sports:tennis}
\end{figure}

As is natural for a game or a sport, this Markov chain has two absorbing
states, ``A wins'' and ``B wins''.%
\index{absorbing states}
Because the victor of the game must win by at least \( 2 \) points,
after \( 4 \) or \( 5 \) transitions the final set of states ``B wins'',
``30:40'', ``30:30'', ``40:30'', ``A wins'' creates an absorbing random
walk on \( 5 \) states.  The exercises compute the probability of
winning a game for specific values of \( p \) and \( q \).

After the first game is over, a second game starts with service passing
to the other side, until a \emph{set} is completed.  This happens when
one side wins at least \( 6 \) games with a margin of at least \( 2 \)
games.  A set is completed as soon as the score is one of 6:0, 6:1, 6:2,
6:3, 6:4, 7;5, 8:6 and so on.  One set follows another until one side
wins the match by winning by at least \( 2 \) sets.  Therefore, a new
Markov chain represents the progress of the set.  The absorbing
probabilities of this higher level chain represent the probability of
winning an individual game.  In this new Markov chain for the set, after
\( 11 \) or \( 12 \) games again there is an absorbing random walk on \(
5 \) states due to the requirement of winning by at least \( 2 \) games.%
\index{absorbing random walk}

Finally, tennis games have a \( 3 \) or \( 5 \) set \emph{match} to
declare the overall winner by at least \( 2 \) sets, making a
third-level Markov chain.  Women and junior players play \( 3 \) set
matches and tennis has additional rules for tiebreakers.

\subsection*{Basketball}

This section assumes familiarity with the rules and terminology of
basketball.%
\index{basketball}
The Markov chain model assumes the states and transition probabilities
of a basketball game between teams A and B are the following.
\begin{description}
    \item[AOff]
        Team A goes on offense.
    \item[APen]
        Penetrating offense by team A.
    \item[AScore]
        Basket scored by team A.
    \item[Bdef]
        Ball in possession of team B in defense.
    \item[BOff]
        Team B goes on offense.
    \item[BPen]
        Penetrating offense by team B.
    \item[BScore]
        Basket scored by team B.
    \item[ADef]
        Ball in possession of team A in defense.
    \item[$p_{\text{eff,A}}$]
        Probability of organizing a penetrating offense by team A.
    \item[$p_{\text{eff,B}}$]
        Probability of organizing a penetrating offense by team B.
    \item[$p_{\text{d,A}}$]
        Probability of successful defense by team A.
    \item[$p_{\text{d,B}}$]
        Probability of successful defense by team B.
    \item[$p_{\text{r,A}}$]
        Probability of offensive rebound by team A.
    \item[$p_{\text{r,B}}$]
        Probability of offensive rebound by team B.
    \item[$p_{\text{s,A}}$]
        Probability of scoring a basket by team A as a result of a
        penetrating offense.
    \item[$p_{\text{s,B}}$]
        Probability of a scoring a basket by team B as a result of a
        penetrating offense.
\end{description}
A diagram of the states and transition probabilities is in Figure~%
\ref{fig:sports:bball}.  This model does not consider three-point shots
and scoring by foul shots.  This simple Markov chain model could be the
basis for extending the model with more states and using advanced sports
analytics for the transition probabilities.

\begin{figure}
    \centering
\begin{asy}
settings.outformat = "pdf";

// import graph;

size(5inches);

real myfontsize = 12;
real mylineskip = 1.2*myfontsize;
pen mypen = fontsize(myfontsize, mylineskip);
defaultpen(mypen);

real h = 1, v = 1, d = 1/2;	// horiz, vert offset spacing
real marge = 1mm;
pair a1 = (0, v), a2 = (h, v), a3 = (2*h, v), a4 = (3*h, v);
pair a5 = (d, 0), a6 = (d+h, 0), a7 = (d+2*h, 0), a8 = (d+3*h, 0);
 
object s1=draw("AOff", ellipse, a1, marge),
   s2=draw("APen", ellipse, a2, marge),
   s3=draw("AScore", ellipse, a3, marge),
   s4=draw("BDef", ellipse, a4, marge);

object s5=draw("BOff", ellipse, a5, marge),
   s6=draw("BPen", ellipse, a6, marge),
   s7=draw("BScore", ellipse, a7, marge),
   s8=draw("Adef", ellipse, a8, marge);

add(new void(picture pic, transform t) {
    draw(pic, Label("$p_{\rm eff,A}$", align=N),
	 point(s1,E,t)--point(s2,W,t), Arrow);
    draw(pic, Label("$1-p_{\rm eff,A}$", align=N),
     	 point(s1,N,t){NNE}..{SSE}point(s4,N,t), Arrow);

    draw(pic, Label("$p_{\rm s,A}$", align=N),
	 point(s2,E,t)--point(s3,W,t), Arrow);
    draw(pic, Label("$1-p_{\rm r,A}-p_{\rm s,A}$", align=1.1*S),
     	 point(s2,ENE,t){NE}..{SE}point(s4,W,t), Arrow);
    draw(pic, Label("$p_{\rm r,A}$", position=Relative(0.25), align=NE),
     	 point(s2,NE,t){NE}..{SE}point(s2,NW,t), Arrow);

   draw(pic, Label("$1$", align=N),
	 point(s3,SW,t)--point(s5,NE,t), Arrow);

   draw(pic, Label("$p_{\rm d,A}$", align=N),
	point(s4,NW,t){NNW}..{SW}point(s1,NNE,t),  Arrow);
   draw(pic, Label("$1-p_{\rm d,A}$", position=Relative(0.25), align=S),
	 point(s4,SW,t)--point(s5,ENE,t), Arrow);
   
   draw(pic, Label("$p_{\rm eff,B}$", align=S),
       point(s5,E,t)..point(s6,W,t), Arrow);
   draw(pic, Label("$1-p_{\rm eff,B}$", position=Relative(0.75), align=N),
	point(s5,E+ENE,t){E}..{SE}point(s8,NW,t), Arrow);

   draw(pic, Label("$p_{\rm s,B}$", align=N),
	point(s6,E,t)--point(s7,W,t), Arrow);
   draw(pic, Label("$1-p_{\rm r,B}-p_{\rm s,B}$", align=1.1S),
     	point(s6,NE,t){ENE}..{SE}point(s8,W,t), Arrow);
   draw(pic, Label("$p_{\rm r,B}$",  position=Relative(0.25), align=SE),
     	point(s6,SE,t){SE}..{NE}point(s6,SW,t), Arrow);

   draw(pic, Label("$1$", align=S),
	point(s7,S,t){S}..{N}point(s1,S,t), Arrow);

   draw(pic, Label("$p_{\rm d,B}$", position=Relative(0.25), align=S),
	point(s8,S,t){SW}..{NW}point(s5,SSE,t), Arrow);
   draw(pic, Label("$1-p_{\rm d,B}$", position=Relative(0.35), align=N),
	point(s8,SSE,t){S}..{N}point(s1,SW,t), Arrow);
});
\end{asy}
    \caption{The state transition diagram for a game of basketball.}%
    \label{fig:sports:bball}
\end{figure}

College basketball games have \( 40 \) minutes playing time for both the
men's and women's leagues.  The games last longer because of time-outs
and foul shots stopping the clock, which this model does not take into
account.  In a typical game, each team will have about \( 70 \)
possessions, measured here as being in the state ``AOff'' and ``Boff'',
because starting in this state leads ultimately to either a score
``AScore'' or a turnover ``BDef''.

As an example, take \( p_{\text{eff,A}} = 0.6 \) and \( p_{\text{eff,B}}
= 0.75 \), so team A is not a good offensively as team B. It is
reasonable to set \( p_{\text{d,A}} = 1 - p_{\text{eff,B}} \) and \( p_{\text
{d,B}} = 1 - p_{\text{eff,A}} \).  Set \( p_{\text{r,A}} = p_{\text{r,B}}
= 0.3 \).  Set \( p_{\text{r,A}} = 0.7 \) and \( p_{\text{r,B}} = 0.6 \)
so that while team A is not as effective on offense, once it gets near
the basket team A scores better than team B near the basket.  Then in
one sample path with \( 450 \) steps in the Markov chain, A had \( 73 \)
possessions and a final score of \( 86 \) and B had \( 74 \) possessions
with a final score of \( 98 \).  Averaging over multiple samples can
give a statistical look at the influence of effective offense versus
basket scoring in winning this simple Markov chain model of basketball.
This is an example where sample path properties are more important than
stationary probabilities.  See the exercises for more details.

\visual{Section Starter Question}{../../../../CommonInformation/Lessons/question_mark.png}
\section*{Section Ending Answer} For tennis, the game points create a
reasonable Markov chain model of a game.  The probability a player can
make a point against as opponent is the analytic measure.  The sets and
matches can be similarly modeled as Markov chains.  For basketball, the
teams are in successive states of offense, near the basket, scoring, and
turnovers.  Sports analytic measures can give the transition
probabilities between states.

\subsection*{Sources} This section is adapted from
\cite{sadovskii93}, Sections 3.2, 3.5, 3.9, 3.10, 3.13.

\hr

\visual{Algorithms, Scripts, Simulations}{../../../../CommonInformation/Lessons/computer.png}
\section*{Algorithms, Scripts, Simulations}

\subsection*{Algorithm}

\begin{algorithm}[H]
    \DontPrintSemicolon
    \KwData{Probability transition matrix \( P \), starting state}
    \KwResult{Either absorption probabilities and waiting time to
      absorption or sample path of the chain}
    \BlankLine
    \emph{Initialization and sample paths}\;
    Load Markov chain library\;
    Set state names, set transition probability matrix, set start
    state\;
    Determine absorption probabilities and waiting times OR\;
    Set an example length and create a sample path of example length\;
    Find sample paths and count scores.\;
\end{algorithm}

\subsection*{Scripts}


\begin{description}

% \item[Geogebra] 

% \link{  .ggb}{GeoGebra applet}

\item[R] 

\link{http://www.math.unl.edu/~sdunbar1/Teaching/Probability/MarkovChains/Sports/tennis.R}{R script for tennis.}

\begin{lstlisting}[language=R]
library("markovchain")

rows <- 17

p <- 0.51
q <- 1-p

P <- matrix(0, rows, rows)
P[1,  2] <- q; P[1,3] <- p;
P[2,  4] <- q; P[2,5] <- p;
P[3,  5] <- q; P[3,6] <- p;
P[4,  7] <- q; P[4,8] <- p;
P[5,  8] <- q; P[5,9] <- p;
P[6,  9] <- q; P[6,10] <- p;
P[7, 16] <- q; P[7,11] <- p;
P[8, 11] <- q; P[8,14] <- p;
P[9, 14] <- q; P[9,12] <- p;
P[10,12] <- q; P[10,17] <- p;
P[11,16] <- q; P[11,13] <- p;
P[12,15] <- q; P[12,17] <- p;
P[13,16] <- q; P[13,14] <- p;
P[14,13] <- q; P[14,15] <- p;
P[15,14] <- q; P[15,17] <- p
P[16,16] <- 1;
P[17,17] <- 1;

tennis  <- new("markovchain", transitionMatrix=P, name="Tennis")
print( summary(tennis) )

cat("Probability A wins game: ",
    absorptionProbabilities(tennis)[1,2], "\n")

T = P[1:15,1:15]
eye = diag(15)
N = solve(eye-T, eye)
ones = matrix(1, 15,1)
VarW = (2*N-eye) %*% N %*% ones - (N %*% ones)^2

cat("Expected length of game in serves: ",
      meanAbsorptionTime(tennis)[1],
    "Standard deviation: ", sqrt(VarW[1]), "\n")
\end{lstlisting}  


\link{http://www.math.unl.edu/~sdunbar1/Teaching/Probability/MarkovChains/Sports/bball.R}{R script for basketball.}
\begin{lstlisting}
library("markovchain")

rows <- 8

peffA <- 0.6
peffB <- 0.75

pdA <-  1-peffB
pdB <- 1 - peffA

prA <- 0.3
prB <- 0.3

psA <- 0.7
psB <- 0.6

stateNames <- c("AOff", "APen", "AScore", "Bdef",
                "BOff", "BPen", "BScore", "ADef")
P <- matrix(c(0,   peffA, 0,   1-peffA,   0,     0, 0, 0,
              0,   prA,   psA, 1-prA-psA, 0,     0, 0, 0,
              0,   0,     0,   0,         1,     0, 0, 0,
              pdA, 0,     0,   0,         1-pdA, 0, 0, 0,
              0,   0,     0,   0,         0,     peffB, 0,   1-peffB,
              0,   0,     0,   0,         0,     prB,   psB, 1-prB-psB,
              1,   0,     0,   0,         0,     0, 0, 0,
              1-pdB,0,    0,   0,         pdB,   0, 0, 0),
              nrow=rows, byrow=TRUE)
rownames(P) <- stateNames
colnames(P) <- stateNames

startState  <- "AOff"

bball  <- new("markovchain", transitionMatrix=P,
                 states=stateNames, name="Basketball")
print( summary(bball) )

pathLength  <- 450
ngames <- 100
allAScores <- numeric(ngames)
allBScores <- numeric(ngames)

for (i in 1:ngames) {
    bballHistory  <- rmarkovchain(n=pathLength, object=bball,
                                 t0=startState)
    allAScores[i] <- 2 * sum(bballHistory == "AScore")
    allBScores[i] <- 2 * sum(bballHistory == "BScore")
}

cat("Mean A Scores: ",      mean(allAScores),
    "Standard Deviation: ", sd(allAScores), "\n")
cat("Mean B Scores: ",      mean(allBScores),
    "Standard Deviation: ", sd(allBScores), "\n")

\end{lstlisting}
% \item[Octave]

% \link{http://www.math.unl.edu/~sdunbar1/    .m}{Octave script for .}

% \begin{lstlisting}[language=Octave]

% \end{lstlisting}

% \item[Perl] 

% \link{http://www.math.unl.edu/~sdunbar1/    .pl}{Perl PDL script for .}

% \begin{lstlisting}[language=Perl]

% \end{lstlisting}

% \item[SciPy] 

% \link{http://www.math.unl.edu/~sdunbar1/    .py}{Scientific Python script for .}

% \begin{lstlisting}[language=Python]

% \end{lstlisting}

\end{description}



% %%% Local Variables:
% %%% TeX-master: t
% %%% End:


\hr

\visual{Problems to Work}{../../../../CommonInformation/Lessons/solveproblems.png}
\section*{Problems to Work for Understanding}
\renewcommand{\theexerciseseries}{}
\renewcommand{\theexercise}{\arabic{exercise}}

\begin{exercise}
    In tennis, assume A has probability \( p = 0.6 \) of winning a
    point.  What is the probability of A winning a game?  If \( p = 0.51
    \), what is the probability of A winning a game?  What is the
    expected length and the standard deviation of the length of the game
    in serves for each of these probabilities?
\end{exercise}
\begin{solution}
    Label the transient states in Figure~%
    \ref{fig:sports:tennis} from \( 1 \) for 0:0 from to bottom, left to
    right and label the absorbing states last.  Then 0:0 is state \( 1 \),
    state 40:30 is state \( 13 \), state 30:40 is state \( 15 \).  The
    absorbing state ``A wins'' is \( 17 \) and ``B wins'' is \( 16 \).
    Use the script with this labeling.

    For \( p = 0.6 \), the probability A wins is \( 0.736 \).  For \( p
    = 0.6 \), the expected number of serves or points until absorption,
    that is the game ends, is \( 6.4842 \) with a standard deviation of \(
    2.590034 \).

    For \( p = 0.51 \), the probability A wins is \( 0.525 \).  For \( p
    = 0.51 \), the expected number of serves or points until absorption,
    that is the game ends, is \( 6.7472 \), only slightly longer.  The
    standard deviation is \( 2.770759 \), also slightly more.

    The conclusion is that most games lasts between \( 4 \) and \( 9 \)
    serves, even if the players are nearly evenly matched.

\end{solution}
\begin{exercise}
    For the Markov chain model of basketball with the given transition
    probabilities, what are the expected scores of the two teams?
\end{exercise}
\begin{solution}
    Using the R script,
\begin{verbatim}
  Mean A Scores:  89.52 Standard Deviation:  7.038825 
  Mean B Scores:  100.54 Standard Deviation:  7.428338 
\end{verbatim}
\end{solution}

\hr

\visual{Books}{../../../../CommonInformation/Lessons/books.png}
\section*{Reading Suggestion:}

\bibliography{../../../../CommonInformation/bibliography}

%   \begin{enumerate}
%     \item
%     \item
%     \item
%   \end{enumerate}

\hr

\visual{Links}{../../../../CommonInformation/Lessons/chainlink.png}
\section*{Outside Readings and Links:}
\begin{enumerate}
    \item
    \item
    \item
    \item
\end{enumerate}

\section*{\solutionsname} \loadSolutions

\hr

\mydisclaim \myfooter

Last modified:  \flastmod

\end{document}

%%% Local Variables:
%%% mode: latex
%%% TeX-master: t
%%% End:
