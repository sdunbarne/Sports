\documentclass[12pt]{article}

\input{../../../../etc/macros}
%\input{../../../../etc/mzlatex_macros}
\input{../../../../etc/pdf_macros}

\bibliographystyle{plain}

\begin{document}

\myheader
\mytitle

\hr

\sectiontitle{Sports as Markov Chains}

\hr

\usefirefox

\hr

% \visual{Study Tip}{../../../../CommonInformation/Lessons/studytip.png}
% \section*{Study Tip}

% \hr

\visual{Rating}{../../../../CommonInformation/Lessons/rating.png}
\section*{Rating} %one of 
% Everyone: contains no mathematics.
Student: contains scenes of mild algebra or calculus that may require guidance.
% Mathematically Mature: may contain mathematics beyond calculus with proofs.
% Mathematicians Only: prolonged scenes of intense rigor.

\hr

\visual{Section Starter Question}{../../../../CommonInformation/Lessons/question_mark.png}
\section*{Section Starter Question}

\hr

\visual{Key Concepts}{../../../../CommonInformation/Lessons/keyconcepts.png}
\section*{Key Concepts}

\begin{enumerate}
  \item 
  \item 
  \item 
\end{enumerate}

\hr

\visual{Vocabulary}{../../../../CommonInformation/Lessons/vocabulary.png}
\section*{Vocabulary}
\begin{enumerate}
  \item 
  \item 
\end{enumerate}

\hr

\visual{Mathematical Ideas}{../../../../CommonInformation/Lessons/mathematicalideas.png}
\section*{Mathematical Ideas}

\subsection*{Tennis}

The scoring of tennis is unusual, for historical reasons which are not
clear.  Roughly speaking, a player's  score advances
by $15$ when a player scores a point because the opponent missed
returning the ball or hits it out of bounds, with a few additional
technical rules.  The maximum score to
win a game is $60$.  So scores should be $0$, $15$, $30$, $45$, and $60$,
except that the third score is abbreviated to $40$, for reasons that
are again historically unclear.

A game consists of a sequence of points played with the same player
serving. A game is won by the first player to have won at least four
points in total and at least two points more than the opponent. The
running score of each game is described in a manner peculiar to
tennis: scores from zero to three points are described as "love",
"15", "30", and "40", respectively. If at least three points have been
scored by each player, making the player's scores equal at 40 apiece,
the score is not called out as "40–40", but rather as "deuce". If at
least three points have been scored by each side and a player has one
more point than the opponent, called ``ad in'' when the
serving player is ahead, and ``ad out''  when the receiving
player is ahead.

To model tennis as a Markov chain, take the states of the chain as the
possible scores.  Assume that transition from one state or score to
another depends only on the present score and not on the preceding
states or scores.  Clearly this assumption is idealizing the game,
leaving out factors such as the order of service, psychological
factors, fatigue, or adaptation to the opponent's style.  Other
factors not mentioned can also affect the transition probabilities.

\subsection*{Basketball}

\visual{Section Starter Question}{../../../../CommonInformation/Lessons/question_mark.png}
\section*{Section Ending Answer}

\subsection*{Sources}
This section is adapted from: 

\nocite{}
\nocite{}

\hr

\visual{Algorithms, Scripts, Simulations}{../../../../CommonInformation/Lessons/computer.png}
\section*{Algorithms, Scripts, Simulations}

\subsection*{Algorithm}

\subsection*{Scripts}

\input{ _scripts}

\hr

\visual{Problems to Work}{../../../../CommonInformation/Lessons/solveproblems.png}
\section*{Problems to Work for Understanding}
\renewcommand{\theexerciseseries}{}
\renewcommand{\theexercise}{\arabic{exercise}}

\begin{exercise}
  
\end{exercise}
\begin{solution}
  
\end{solution}
\begin{exercise}
  \begin{enumerate}[label=(\alpha*)]
  \item 
  \end{enumerate}
\end{exercise}
\begin{solution}
  \begin{enumerate}[label=(\alpha*)]
  \item 
  \end{enumerate}
\end{solution}

\hr

\visual{Books}{../../../../CommonInformation/Lessons/books.png}
\section*{Reading Suggestion:}

\bibliography{../../../../CommonInformation/bibliography}

%   \begin{enumerate}
%     \item 
%     \item 
%     \item 
%   \end{enumerate}

\hr

\visual{Links}{../../../../CommonInformation/Lessons/chainlink.png}
\section*{Outside Readings and Links:}
\begin{enumerate}
  \item  
  \item  
  \item  
  \item 
\end{enumerate}

\section*{\solutionsname}
\loadSolutions

\hr

\mydisclaim \myfooter

Last modified:  \flastmod

\end{document}

%%% Local Variables:
%%% mode: latex
%%% TeX-master: t
%%% End:
